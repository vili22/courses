\documentclass[aps,prb,10pt,twocolumn,groupedaddress]{revtex4-1}
%\setlength\topmargin{4.6mm}
%\documentclass{[prl,twocolumn]{revtex4-1}
%\usepackage[ansinew]{inputenc}
%\usepackage[latin1]{inputenc}% PROPER ENCODINGS
%\usepackage[T1]{fontenc}%      FOR FINNISH TEXT
%\usepackage[finnish]{babel}% 
%\usepackage[official]{eurosym}

%\usepackage{subfig}
\usepackage{caption}
\usepackage{subcaption}
\usepackage{graphicx}
\usepackage{epsfig}
\usepackage{epstopdf}
\usepackage{amsmath}
\input{mlbp17_macros}

%\usepackage{subfig}
%\usepackage[footnotesize]{caption}
%\pagestyle{empty}
%\setlength{\textwidth}{140mm}
%\setlength{\textheight}{240mm}
%\setlength{\parindent}{0mm}
%\setlength{\parskip}{3mm plus0.5mm minus0.5mm}
\bibliographystyle{apsrev4-1}

%%%%%%%%%%%%%%%%%%%%%%%%%%%%%%%%%%
\begin{document}

\title{Logistic regression and Bayes-classifier study of classification of
  songs to genres based on timbre, pitch and rhythm of the music signal }
\date{\today}
\author{Ville Virkkala}
\affiliation{Aalto University, P.O. Box 11100,
FI-00076 Aalto, Finland}

\begin{abstract}
  A genre of a song can be estimated based on its music signal's
  characteristics . In this work we use two linear classifiers, Bayes Classifer
  and Logistic classifer, to classify songs into one of ten possible genres. The
  two classifers are trained against the training data and their performance is
  compared against each other. In this work we show that both classifiers
  perform much better compared to random guess. However their cabability to
  classify all songs is clearly limited the accuracy for both classifiers being
  around 60\%.
\end{abstract}

\maketitle

\section{Introduction}
An automatic music transcription, \textit{i.e.}, notating a piece of music to
a speficic genre, \textit{e.g.}, Blues, dates back to 1970s when first attempts
towards automatic music transcription were made\cite{musictranscription}. Since
then interest in
automatic transcription of music has grown rapidly and various approaches,
statistical methods, modelling human auditory system, have been applied to
music transcription problem. However even today an expert human musician often
beats a state-of-the-art automatic transcription system in accuracy.

Characteristics of music signal that are useful in classification of a song
are \textit{timbre}, \textit{rhytm}, \textit{pitch}, \textit{loudness} and
\textit{duration}\cite{musictranscription} from which the three first one,
descriped below are used in this work.
\begin{itemize}  
\item The timbre of the music can be most easily described as the factor
  which separates two sources of music from each other. For example if the same
  song is played by violin or a quitar the timbre is called the characted which
  separates the violin from the quitar.
\item The pitch is related to frequency scale of a song a can be defined as the
  frequency of the sine-wave fitted to target sound by human listener.
\item The rhythm of the music can be described as arrangement of sounds as time
  flows.
\end{itemize}

In classification problem the object is classified into a certain class based
on it's characteristics called features. A linear classifier does the
classification by making a linear combination of the features and converting
the resulting value into a class or a probability that the object belongs to
given class. In logistic regression the feature vector of the object is
transformed into a probality by taking a linear compination of features and
mapping the result into interval $\lbrack 0, 1\rbrack$ using a sigmoid function.
The Bayes-classifier incontrast assumes that the feature vector is drawn from a
multidimensional-gaussian distribution. The posterior probalility of the object
belonging to a certain class is then obtained as a product of the prior of the
class and the probability of to sample the given feature vector from it
multidimensional gaussian distribution.

The paper is organized as follows. The used data-set and the computational
methods are described in detail in Sec. \ref{sec:methods}. In Sec.
\ref{sec:results} the results for the both logistic regression- and
Bayes-classifier are given. Sec. \ref{sec:conclusions} is a
summary of the results and the differences between the two classfiers are
discusssed.

\section{Used data-set and computational methods}
\label{sec:methods}
\subsection{Used data-set}
\label{sec:used_data_set}
The data-set consisted of 4363 songs and was divided into training and test data
sets including every third song to test set and rest of the songs to training
set. Each song contained 264 features and the songs were labeled to 10
different categories. The gatecories were: 1 Pop Rock, 2 Electronic, 3 Rap,
4 jazz, 5 Latin, 6 RnB, 7 International, 8 Country, 9 Reggae and 10 Blues.
The musical characteristics of the songs were packed to a feature
vector of length 256. The first 48 elements in the feature vector can be
associated to timbre, the next 48 elements to pitch and the final 168 features
to rhythm. The distribution of the features resembled in most cases a gaussian
distribution or a skew symmetric distribution. This is illustrated figures
\ref{fig:feature_distribution}a and \ref{fig:feature_distribution}b.
\begin{figure}[!t]
  \centering
  \begin{subfigure}[]{0.38\textwidth}
    \centering
    \includegraphics[width=\textwidth]{feature_gaussian.eps}
    \caption{}
  \end{subfigure}\\
  %\vspace{0.1cm}
  \centering
  \begin{subfigure}[]{0.38\textwidth}
    \centering
    \includegraphics[width=\textwidth]{feature_skew_symmetric.eps}
    \caption{}
  \end{subfigure}\\
  \caption{Visualization of typical distributions of features, gaussian distribution (a) and skew-symmetric distribution (b).}
  \label{fig:feature_distribution}
\end{figure}

\subsection{Computational methods}
\label{sec:computational_methods}
In this work two different methods were used to classify the songs to different
genres. First method is the gradient descent method in which the logistic-loss
is minimized iteratively using the gradient descent method.
The other method used is the Bayes-classifier which classifies the song to
certain gategory that gives the maximum posterior probability with respect to
label $i$. Both methods are described below in detail. In addition we studied
the effect of feature extraction and for that purpose we used principal
component analysis method to exlude features with little impact. 
\subsubsection{Gradient descent method}
\label{sec:gradient_descent}
In logistic regression for binary classifier problem the starting point is the
minimization of the loss function
\vspace*{-2mm}
 $$\emperror((\vx,y);\vw) =\min\limits_{\vw \in \mathbf{R}^{2}} (1/\samplesize)\sum_{\sampleidx=1}^{\samplesize} L((\vx^{(\sampleidx)},y^{(\sampleidx)}); \vw),$$
where the logistic loss $L((\vx,y); \vw)$ is defined as   $L((\vx,y); \vw) = \ln\big(1 + \exp\big(- y (\vw^{T} \vx)\big))\big)$ and
$\vx$ is the feature vector of a music signal, $\vw$ are the coefficients of the linear expansion and $y$ is the label 1 or -1 whether the
song belongs to certain category or not. The minimization problem can be further converted to
\begin{eqnarray}
  \emperror((\vx,y);\vw)&=&\frac{1}{N}\max\limits_{\vw \in \mathbf{R}^{2}}\sum_{i=1}^{N}\mathrm{ln}(\frac{1}{1+\mathrm{exp}(-y^i(\vw^T\vx))})\nonumber\\
  &=&\frac{1}{N}\max\limits_{\vw \in \mathbf{R}^{2}}(\sum_{y^i=1}\mathrm{ln}(\frac{1}{1+\mathrm{exp}(-\vw^T\vx^{(y^i)})})\nonumber\\
  & & + \sum_{y^i=-1}\mathrm{ln}(1-\frac{1}{1 + \mathrm{exp}(-\vw^T\vx^{(y^i)})}))\\
  &=&\max\limits_{\vw \in \mathbf{R}^{2}}(\sum_{y^i=1}\mathrm{ln}(p_{y^i=1})\nonumber\\
  & & + \sum_{y^i=-1}\mathrm{ln}(1-p_{y^i=1})),
  \label{eq:logreg1}
\end{eqnarray}
where $p_{y^i=1}$ is the probability that the song $i$ is labeled belonging to certain category.
There is no closed form solution for equation (\ref{eq:logreg1}) and for that reason some numberical iterative solver must
be used to find the optimal $\vw$. One of the most popular methods to find the optimal solution is gradient descent (GD) method.
In GD the weights $\vw$ are updated at each iteration $\mathrm{k+1}$ according to equation
\begin{equation}
  \vw^{(k+1)} = \vw^{(k)} - \alpha \nabla \emperror(\mathbf{w}^{(k)}).
  \label{eq:gd}
\end{equation}

To be able to use the GD method we need know the gradients  $\nabla \emperror(\mathbf{w}^{(k)})$.
By marking $t^i=\mathrm{max}(0,y^i)$ we can write the equation (\ref{eq:logreg1}) as
\begin{eqnarray}
  \emperror((\vx,y);\vw)&=&\min\limits_{\vw \in \mathbf{R}^{2}}-\sum_i^N(t^i\mathrm{ln}(p_{y^i})\nonumber\\
  & &+(1-t^i)\mathrm{ln}(1-p_{y^i})).
  \label{eq:logreg2}
\end{eqnarray}
The derivative of $p_{y^i}$ with respect to $w_j$ can be written as $\frac{\partial p_{y^i}}{\partial w_j}=\frac{\partial p_{y^i}}{\partial (\vw\vx)}\frac{\partial (\vw\vx)}{\partial w_j} = p_{y^i}(1-p_{y^i})x_j^i$. Thus taking derivative of equation (\ref{eq:logreg2}) with
respect to $w_j$ we get for $\nabla \emperror(w_j^k)$.
\begin{eqnarray}
  \nabla \emperror(w_j^k) &=&-\sum_i(t^i\frac{1}{p_{y^i}}(1-p_{y^i})p_{y^i}x_j^i\nonumber\\
  & &+(1-t^i)\frac{1}{1-p_{y^i}}(-1)p_{y^i}(1-p_{y^i})x_j^i)\nonumber\\
  &=&-\sum_i\left(t^i -t^ip_{y^i}-p_{y^i}+t^ip_{y^i}\right)x_j^i\nonumber\\
  &=&-\sum_i(t^i-p_{y^i})x_j^i.
\end{eqnarray}
Now the $\nabla \emperror(w_j^k)$'s can be written in vector form as
\begin{equation}
  \nabla \emperror(\mathbf{w}^k) = -\left(\hat{\vy}-\frac{1}{1+\mathrm{exp}(-(\vw^{k})^T X^T)}\right)^TX,
  \label{eq:gd_gradients}
\end{equation}
where $\hat{\vy} = \mathrm{max}(\mathbf{0},\vy)$ is a vector of length $N$ for which the
$i$-th element is zero if $y_i=-1$ and one if $y_i$ = 1. The matrix $X$ is a
matrix which rows are the feature vectors $\vx^i$. Thus we can solve the minimization
problem (\ref{eq:logreg2}) iteratively using the GD method (\ref{eq:gd}) and the gradients (\ref{eq:gd_gradients}).
\subsubsection{Bayes classifer}
\label{sec:bayes_classifier}
For a Bayes classifier we assume that the distribution of the feature vector of
a music signal with respect to label $y_i$ is a Gaussian distribution
\begin{equation}
  p(\vx|y_i;\mathbf{m}_{i},\mathbf{C}_{i}) = \frac{1}{\sqrt{\det\{ 2 \pi \mathbf{C}_{i} \}}}e^{-(1/2) (\vx\!-\!\vm_{i})^{T} \mathbf{C}_{i}^{-1} (\vx\!-\!\vm_{i})}.
  \label{eq:bayes_gaussian_distribution}
\end{equation}
Using the Baye's theorem the posterior probability
$p(y_i|\vx;\mathbf{m}_{i},\mathbf{C}_{i})$ can be written as
\begin{equation}
  p(y_i|\vx;\mathbf{m}_{i},\mathbf{C}_{i}) = \frac{p(y_i)p(\vx|y_i;\mathbf{m}_{i},\mathbf{C}_{i})}{p(\vx)},
  \label{eq:bayes_posterior}
\end{equation}
where the $p(\vx)$ is a normalization constant and can be omitted. To be able to
use the equation (\ref{eq:bayes_posterior}) we need to find optimal values for
parameters $p(y_i)$, $\vm_i$ and $\mathbf{C}_i$. The prior $p(y_i)$ can be
simply estimated as the fraction of labels $y_i$ among all labels.
Because the samples ${\vx^{(t)}, y^{(t)}}$ are independent the parameters
$\vm_i$ and $\mathbf{C}_i$ and can be obtained by
maximizing the respective log-likelihood function with respect to the
parameters. The log-likelihood
can be written as
\begin{eqnarray}
  \mathcal{L}_{y=i}&=&\sum_{y^t=i}\mathrm{log}(P(\vx^t|y^t=i;\vm_i,\mathbf{C}_i))\nonumber\\
  &=&\sum_{y^t=i}(-\frac{1}{2}\mathrm{log}(2\pi^n)-\frac{1}{2}\mathrm{log}(\mathrm{det}(\mathbf{C}_i))\nonumber\\
  & &-\frac{1}{2}(\vx^t-\vm_i)^T\mathbf{C}_i^{-1}(\vx^t-\vm_i)).
\end{eqnarray}
Now the optimal $\vm_i$ is obtained by setting the derivative of
$\mathcal{L}_{y=i}$ with respect to $\vm_i$ to zero, \textit{i.e.},
\begin{eqnarray}
  \frac{\partial{\mathcal{L}_{y=i}}}{\partial\vm_i}&=&\mathbf{C}_i^{-1}\sum_{y^t=i}(\vx^t-\vm_i).
  \label{eq:bayes_ms}
\end{eqnarray}
Setting eq. (\ref{eq:bayes_ms}) to zero and solving for $\vm_i$ gives
$\vm_i=\frac{\sum_{y^t=i}\vx^t}{N}$. Similarly the
$\mathbf{C}_i$ is obtained  by setting the derivative of $\mathcal{L}_{y=i}$ with respect to $\mathbf{C}_i$ to zero giving
\begin{eqnarray}
  \frac{\partial{\mathcal{L}_{y=i}}}{\partial\mathbf{C}_i}&=&\sum_{y^t=1}(-\frac{1}{2}\mathbf{C}_i^{(-1)}\nonumber\\
  & &+\frac{1}{2}\mathbf{C}_i^{(-1)}(\vx^t-\vm_i)(\vx^t-\vm_i)^T\mathbf{C}_i^{-1}).
  \label{eq:bayes_cs}
\end{eqnarray}
Setting eq. (\ref{eq:bayes_cs}) to zero and solving for $\mathbf{C}_i$ gives
$\mathbf{C}_i=\frac{1}{N}\sum_{y^t=i}(\vx^t-\vm_i)(\vx^t-\vm_i)^T$. In equations
(\ref{eq:bayes_ms}) and (\ref{eq:bayes_cs}) the identieties
$\frac{\partial(\vx-\vs)^T\mathbf{W}(\vx-\vs)}{\partial\vs} = 2\mathbf{W}(\vx-\vs)$, $\frac{\partial\mathrm{ln}|\mathrm{det}(\mathbf{X})|}{\partial\mathbf{X}}=(\mathbf{X}^T)^{-1}$ and  $\frac{\partial\mathbf{a}^T\mathbf{X}^{-1}\vb}{\partial\mathbf{X}}=-\mathbf{X}^{-T}\mathbf{a}\vb^T\mathbf{X}^{-T}$ from matrix cook-book
\cite{matrixcookbook}
were used. In classification the parameters $p(y_i)$, $\vm_i$ and $\mathbf{C}_i$
are optimized for all ten classes using the training data. The song is then
classified to certain category $i$ that maximizes the posterior probability
(\ref{eq:bayes_posterior}), \textit{i.e.},
\begin{equation}
  i = \argmax_i p(y_i|\vx;\mathbf{m}_{i},\mathbf{C}_{i}).
\end{equation}
\subsubsection{Principal component analysis}
\label{sec:principal_component}
Let $\mathbf{X}$ be matrix which rows are the feature vectors, \textit{i.e.},
the dimension of the matrix is $n\times p$ where $n$ is the number of samples
and p is the length of the feature vector.
The sample covariance matrix $\mathbf{C}$ is then obtained as
$\mathbf{C}=\mathbf{X}^T\mathbf{X}/(n-1)$ which can be diagonalized as
\begin{equation}
  \mathbf{C}=\mathbf{V}\mathbf{L}\mathbf{V}^T,
  \label{eq:c_diagonalization}
\end{equation}
where $\mathbf{V}$ are the eigenvectors of $\mathbf{C}$ and are called
the principal axes.
The matrix $\mathbf{X}$ can be decomposed using singular-value decomposition as
$\mathbf{X}=\mathbf{U}\mathbf{S}\mathbf{V}^T$, where $\mathbf{S}$ is a diagonal
matrix containing the singular values of $\mathbf{X}$. Now the matrix
$\mathbf{C}$ can be written as
\begin{equation}
  \mathbf{C} = \mathbf{V}\frac{\mathbf{S}^2}{n-1}\mathbf{V}^T.
  \label{eq:c_svd}
\end{equation}
Comparing equation (\ref{eq:c_svd}) to (\ref{eq:c_diagonalization}) shows that
the principal axes are the same as the right singular vectors of $\mathbf{X}$.
Now the principal components of $\mathbf{X}$ are obtained as
$\mathbf{X}\mathbf{V}=\mathbf{U}\mathbf{S}$. Now we can include only feature
vectors that have impact by exluding components that corresponds to singular
values below some threshold. In this work we included singular values of which
sum contained 90\% of the total sum of the singular values and exluded rest.
\section{Results}
\label{sec:results}
\begin{figure}[!t]
  \centering
  \includegraphics[width=0.42\textwidth]{Ga_256As_255N.eps}
  \caption{(Color online) Band structure of the periodic Ga$_{256}$As$_{255}$N 
system. The energy zero coincides with the top of the valence band. The
vertical lines are guide to an eye and indicates the position of the 
high-symmetry points.}
  \label{fig:band_structure}
\end{figure} 

SC-DFT calculations model alloys through band structures corresponding to 
the superlattice periodicity. Although this is an artificial approach for a
random alloy it gives valuable first-principles information about
the N-host  and N-N interactions. Figure \ref{fig:band_structure} shows
the band structure of the periodic Ga$_{256}$As$_{255}$N system
in the small first Brillouin zone corresponding to the large unit cell of 512
atoms. The lowest conduction band, {\it i.e.}, the CBE is 
flat outside the central region and bends down when approaching the
$\Gamma$ point. The flat region is due to the localized nitrogen-induced 
resonant states that hybridizes especially near the $\Gamma$ point with GaAs 
bulk states. Anticipating our discussion below the strong dispersion near the 
$\Gamma$ point can also be interpreted to originate from long-range 
nitrogen-induced resonant states interacting with each other.


The nitrogen-induced states result in characteristic features in the
LDOS's. Fig. \ref{fig:ldos}a shows the LDOS's
for Ga$_{256}$As$_{255}$N, Ga$_{108}$As$_{106}$N and Ga$_{32}$As$_{31}$N systems. 
LDOS's have a narrow peak and a low intensity tail
toward lower energies corresponding to the flat and dispersive regions
of the CBE, see Fig. 1, respectively. We find these features also in the case 
of GaP$_{1-x}$N$_x$ as is evident in Fig. \ref{fig:ldos}b for 
Ga$_{256}$P$_{255}$N, Ga$_{108}$P$_{107}$N and Ga$_{32}$P$_{31}$N systems. 
In comparison with GaAs host the LDOS peak related to nitrogen-induced states 
in GaP is at the same N concentrations lower in energy with respect to the 
CBM and the extent of the LDOS tail
is clearly smaller.  When the size of the SC is reduced from 512
to 216 and further to 64 atoms the peak corresponding to nitrogen-induced 
states broadens in both alloys but stays stationary 
with respect to the top of the valence band and the LDOS tail reaches deeper 
in to the band gap causing its reduction. This behavior is in a good qualitative
agreement with the behavior of the A line in photoluminescence spectra for 
GaP$_{1-x}$N$_x$ alloys with increasing N concentration.\cite{yaguchi}

The reduction of the band gap and the length of the LDOS tail below the peak
corresponding to nitrogen-induced states for the GaAs and GaP SC's with a 
single N atom 
and of different sizes are given in Table \ref{tab:concentrations}. For GaP the
band gap reduction is in practice the same as the length of the LDOS tail 
reflecting the position of the nitrogen-induced states just at 
the bulk CBM at the $\Gamma$-point. In the case of GaAs
a bias of about 0.5 - 0.6 eV should be subtracted from the length of the LDOS 
tail to obtain the band gap reduction. The bias is due to the location of the 
peak corresponding to nitrogen-induced resonant states within the bulk 
conduction band. The lengthening of the LDOS tail 
causing the band gap reduction is a common feature for the GaAs$_{1-x}$N$_x$ and 
GaP$_{1-x}$N$_x$ systems calling for a uniform picture for these materials.

\begin{center}
  \begin{table}[t]
    \caption{Band gap reductions and LDOS tail lengths (See Fig. 2) 
    for different periodic GaAs$_{1-x}$N$_x$ and GaP$_{1-x}$N$_x$ systems. The
band gap reduction is determined at the $\Gamma$ point of the superlattice.}
    \begin{tabular*}{0.45\textwidth}{@{\extracolsep{\fill}}lcc}
      \hline
      \hline
      System SC  & $\Delta E_g$ (eV) & Tail length (eV)\\
      \hline
      Ga$_{256}$As$_{255}$N & 0.02 & 0.53\\ 
      Ga$_{108}$As$_{107}$N & 0.09 & 0.67\\
      Ga$_{32}$As$_{31}$N & 0.34 & 0.99\\
      \hline
      Ga$_{256}$P$_{255}$N & 0.13 & 0.13\\
      Ga$_{108}$P$_{107}$N & 0.22 & 0.21\\
      Ga$_{32}$P$_{31}$N & 0.57 & 0.56\\
      \hline
      \hline
    \end{tabular*}
    \label{tab:concentrations}
  \end{table}
\end{center}
Figure \ref{fig:cbe_3D} shows the CBE partial charge density in GaAs
around a N atom as a density isosurface. In GaP we find a similar
behavior. The partial charge density is  
obtained by summing the densities from all the k-points used in the calculation.
The SC used contains 512 atoms and the isosurface is viewed along a 
$\langle$111$\rangle$ direction 
so that
\begin{figure}[!b]
  \centering
    \includegraphics[width=0.36\textwidth]{cbe_3D.eps}
    \caption{(Color online). CBE partial charge density in the
Ga$_{256}$As$_{255}$N SC, viewed along the $\langle$111$\rangle$
direction. The isosurface shown corresponds to a density isovalue of 0.0012.
The N atom is located at the center of the figure.}
  \label{fig:cbe_3D}
\end{figure}a Ga atom is on top of the N atom in the  center of the figure.
The density agglomerates strongly around the N atom and stretches
out toward the four nearest-neighbor Ga atoms and the N-Ga back bonds. 
Thereafter, the agglomeration of the CBE partial charge density is directed
toward the 12 zigzag directions (six of them are on a \{111\} plane 
perpendicular
to the direction of view, three of them point upwards and three downwards
from that \{111\} plane). We have checked that the CBE partial density from 
all the k-points, including those with dispersion and near 
the $\Gamma$ point, have this anisotropic character. The interaction 
between the nitrogen-induced states takes place along the the zigzag 
chains
with  the CBE partial charge agglomeration. Thus, when the N concentration 
increases both the width of the narrow LDOS peak and the extent of the
LDOS tail increase simultaneously with the strengthening of the CBE 
partial charge agglomeration on the connecting zigzag chains.

The tendency of the CBE charge to localize along the zigzag chains
is strongly connected with ionic relaxations, initiated by the strong and short 
Ga-N bonds also propagating along these chains. According to our SC-DFT
calculations the Ga-N bond attains in GaAs and GaP nearly the same value as in 
bulk GaN resulting in an inward relaxation of the nearest neighbor Ga  atoms by 
13-16 \% of the bond length of the GaAs and GaP lattices. A similar strong
ionic relaxation along the zigzag directions has been previously observed
in the case of the vacancy in silicon \cite{wang_cz} and a
similar anisotropy of localized N derived states in GaAs and GaP was
predicted by Kent and Zunger.\cite{zunger1} We can see the
agglomeration of the CBE partial charge density also when we omit the 
ionic relaxation from ideal lattice
positions but then the effect is clearly weaker and shorter in range. This 
is  in agreement with Kent and Zunger \cite{zunger1} who
predicted a  stronger band gap reduction when the ionic structures were relaxed 
in SC  calculations.
\begin{center}
  \begin{table}[!t]
    \caption{Band gap reduction due to a single N atom or two N atoms
      in the 512- and 216-atom  GaAs SC's. Three different
      configurations are considered for two N atoms.}
    \begin{tabular*}{0.4\textwidth}{@{\extracolsep{\fill}}lc}
      \hline \hline Configuration & $\Delta E_g$ (eV)
      \\ \hline 
         Ga$_{256}$As$_{255}$N                  & 0.02
      \\ Ga$_{256}$As$_{254}$N$_2$ [(1,1,0)$a$] & 0.09
      \\ Ga$_{256}$As$_{254}$N$_2$ [(1,1,1)$a$] & 0.06
      \\ Ga$_{256}$As$_{254}$N$_2$ [(1,0,0)$a$] & 0.04
      \\ Ga$_{108}$As$_{107}$N                  & 0.09
      \\ Ga$_{108}$As$_{106}$N$_2$ [(1,1,0)$a$] & 0.20
      \\ Ga$_{108}$As$_{106}$N$_2$ [(1,1,1)$a$] & 0.17
      \\ Ga$_{108}$As$_{106}$N$_2$ [(1,0,0)$a$] & 0.12 
      \\ Ga$_{32}$As$_{31}$N                   & 0.34
      \\ \hline
      \hline
    \end{tabular*}
    \label{tab:configurations}
  \end{table}
\end{center}

We have studied the anisotropy and strength of the N-N interaction
also by inserting two N atoms at different positions in GaAs
SC's of 216 and 512 atoms. The studied representative
configurations contain one N atom in the origin and another
one in the (1,1,1)$a$, (1,1,0)$a$, or (1,0,0)$a$ location, where $a$
is the lattice parameter of the conventional unit cell. The results are given
in Table \ref{tab:configurations} in terms of the reduction of the band gap with
respect to the GaAs bulk band gap. The behavior of the band gap is
qualitatively similar for the two SC sizes signalizing
from the interaction between the N atoms inside the same SC.
The strongest reduction in the band gap is observed for the (1,1,0)$a$
configuration in accordance with the above discussion about the
anisotropy of the nitrogen-induced resonant states.
In the (1,0,0)$a$ configuration the N atoms are not along the same
zigzag chain and consistently the band gap reduction is modest compared to the
case of a single N atom in the SC. In the (1,1,1)$a$ configuration the band gap
reduction is intermediate between those of the (1,1,0)$a$ and (1,0,0)$a$
configurations. This  is due to the fact that the strong back bond at a
Ga atom caused by one of  the N atoms is next to the other N atom.

In order to study the effect of the interactions between the nitrogen-induced 
states on the band gap reduction in random structures in accordance 
with the
experimental conditions we have developed on the basis of our \textit{ab initio}
results a TB model, in which only interactions between N atom sites
connected through zigzag chains are included. In our model the non-diagonal
matrix elements $h_{i,j}$, describing the interaction between the N atom sites
$i$ and $j$, are defined as $h_{i,j}=k/r_{i,j}^{\alpha}$ if the sites are connected 
through a zigzag chain and $h_{i,j} = 0$ otherwise. Here $r_{i,j}$ is the distance
between the two N atoms. The power-law decay reflects the long-range tendency of
the directional interaction. The diagonal terms $h_{i,i}$ are set to a 
constant value describing the energy level of the nitrogen-induced states. 
To determine the
parameters $k$ and $\alpha$ we used our LDOS results calculated within the 
SC-DFT scheme for four large 
structures of a single N atom in the SC's of 64-, 216-, 512-atoms 
(Fig. \ref{fig:ldos}) and of two N 
atoms at the nearest anion sites in the 216-atom SC 
(Fig. \ref{fig:ldos_N2_near}). The parameter values of
$k=-0.67$ eV\r{A}$^\alpha$, $\alpha=1.28$ reproduce
the DFT LDOS peak and tail structures (See Fig. \ref{fig:ldos}) of N in GaAs
and $k=-0.59$ eV\r{A}$^\alpha$, $\alpha=1.43$ those of N in GaP. The 
implementation of the developed method is explained in more detail in 
Appendix \ref{appendix}.
\begin{figure}[!b]
  \centering
  \includegraphics[width=0.42\textwidth]{ldos_plots_N2_near.eps}
  \caption{(Color online) First-principles LDOS's corresponding to two N atoms
at neighboring anion sites in Ga$_{108}$As$_{106}$N$_2$ and
Ga$_{108}$P$_{106}$N$_2$ systems. The solid segments indicate the distance between
the two peaks related to nitrogen-induced states and the distance between the 
center of mass of the two peaks and the minimum eigenvalue. The energy zero 
coincides with the top of the valence band.}
  \label{fig:ldos_N2_near}
\end{figure}

Using our \textit{ab initio} -based TB model we study changes 
in the CBE as a function of the N concentration. We randomly distribute from 
384 up to 13,824 N atoms into a SC of 442,368 anion sites. The resulting 
Hamiltonian matrix is diagonalized
and a broadening eigenvalue distribution around the original energy level of
nitrogen-induced states is obtained. Between the extreme values there exists a 
distribution of eigenvalues
corresponding to the continuous broadening of the nitrogen-induced states seen
in the 
LDOS's. Fig. \ref{fig:random_dist}  shows as an example the distribution for a
large GaAs$_{1-x}$N$_x$ sample with a N concentration of 3.1\%. The highest peak
is due to the isolated N atoms and the two lower ones are related to N-N pairs.
These features are in agreement with the measured scanning tunneling spectra
.\cite{ivanova}
\begin{figure}[!t]
  \centering
  \includegraphics[width=0.42\textwidth]{random_dist.eps}
  \caption{(Color online) TB eigenvalue distribution 
corresponding to a random sample of 13,824 N atoms (3.1\% concentration)
in the GaAs$_{1-x}$N$_x$ alloy.}
  \label{fig:random_dist}
\end{figure} 

\begin{figure}[!t]
  \centering
  \begin{subfigure}[]{0.38\textwidth}
    \centering
    \includegraphics[width=\textwidth]{conc_dep_GaAs.eps}
    \caption{}
  \end{subfigure}\\
  \vspace{0.2cm}
  \centering
  \begin{subfigure}[]{0.38\textwidth}
    \centering
    \includegraphics[width=\textwidth]{conc_dep_GaP.eps}
    \caption{}
  \end{subfigure}
  \caption{(Color online) Broadening of the distribution of nitrogen-induced 
states as a function of N concentration in (a) GaAs$_{1-x}$N$_x$  and (b) 
GaP$_{1-x}$N$_x$. Blue circles and 
squares give our random-system TB and SC-DFT results, respectively. For 
GaAs$_{1-x}$N$_x$ red triangles are experimental data from Ref. 
\onlinecite{klar} (measurement temperature 300  K), whereas for  
GaP$_{1-x}$N$_x$  red upright and downright triangles  are experimental data 
from Refs. \onlinecite{yaguchi} (20 K) and \onlinecite{shan2} 
(room temperature),
respectively. The energy zero is the energy level corresponding to isolated
nitrogen-induced states (In calculations the position of the peak corresponding
to nitrogen-induced states in LDOS). To align the experimental and calculated 
energy levels corresponding to nitrogen-induced states in GaAs$_{1-x}$N$_x$ the 
experimental data is shifted to locate symmetrically with respect to the energy 
zero.}
\label{fig:splitting}
\end{figure} 
We plot, respectively, in Figs. \ref{fig:splitting}a and 
\ref{fig:splitting}b for GaAs$_{1-x}$N$_x$ and GaP$_{1-x}$N$_x$ for each
N concentration the minimum and maximum values of the eigenvalue distribution.
The values are averaged over 100 different random samples. 
The extreme values show a square root like behavior, which is, in the case 
GaAs$_{1-x}$N$_x$, in a good agreement with photomodulated reflectance 
measurements for the N-induced band gap reduction and a N-induced feature in the
conduction band.\cite{klar} Similarly, for GaP$_{1-x}$N$_x$  our results are in 
good agreement with the band gap reduction measured by 
photoluminescence\cite{yaguchi} and photomodulated transmission 
spectroscopy.\cite{shan2} Our 
present SC-DFT results shown in Fig. 4 as well earlier SC-DFT
calculations\cite{zunger1,virkkala} give a linear dependence of the band gap 
reduction on the N concentration which is in a clear disagreement with 
experimental findings. The reason for the linear dependence is the fact that N 
atoms of the neighboring SC's are on the same zigzag chains resulting in a 
surplus of N-N interactions with relatively short distances in comparison with
the random N atom distribution of the same concentration.

\section{Conclusions}
\label{sec:conclusions}
In this work we demonstrated using \textit{ab initio} calculations 
how the nitrogen-induced states near the CBM propagate along zigzag chains in 
GaAs$_{1-x}$N$_x$ and GaP$_{1-x}$N$_x$ alloys. This results in coupling between
states originating from different N atoms which becomes stronger with 
increasing N concentration leading to the broadening of the distribution of 
nitrogen-induced states. On the basis of our DFT results we constructed a TB 
model for the interaction of the nitrogen-induced 
states and applied it in large random
systems of N atoms in GaAs and GaP. The model predicts a square-root-like
broadening of the distributions of nitrogen-induced states as a 
function of the N 
concentration and a corresponding narrowing of the band gap in agreement with
experiments. The square-root-like behavior is due to the interplay between the
directional and long-range characters of the interactions between the 
nitrogen-induced states. Thus, the band gap 
narrowing in dilute III-V nitrides can be qualitatively and quantitatively 
explained by \textit{ab initio} calculations, and it is an 
inherent property of the interactions between nitrogen-induced states 
mediated by the host lattice, rather than nitrogen host material CBE
interaction.

\section{Acknowledgments}
We acknowledge the Suomen Kulttuurirahasto foundation
for financial support. This work has been supported by
the Academy of Finland through the Center of Excellence program. 
The computer time was provided by CSC -- the Finnish IT Center
for Science.

\appendix*
\section{Developed TB model}
\label{appendix}
We have developed a TB model describing the interaction between
nitrogen-induced states originating from N atoms substituting anions 
in a III-V compound semiconductors. The non-diagonal matrix elements 
$h_{i,j}$ corresponding to the N atoms $i$ and $j$, are 
defined as $h_{i,j}=k/r_{i,j}^{\alpha}$ if atoms $i$ and $j$, separated by the 
distance $r_{i,j}$, are connected trough a zigzag chain and $h_{i,j}=0$ otherwise.
The diagonal terms are set to an arbitrary chosen 
constant value $E_{s^*}$ describing the energy level of the isolated 
nitrogen-induced 
states. The units are electron volts and angstroms. Using the supercell 
approximation with periodic boundary conditions and the $\Gamma$-point 
approximation the non-zero matrix elements $h_{i,j}$ become
\begin{equation}
  h_{i,j}=\sum_{\mathbf{L}}\frac{k}{\left|\mathbf{\hat{r}}_{i,j}+\mathbf{L}\right|^{\alpha}},
\label{eq:periodic1}
\end{equation}
where $\mathbf{L}$ is a vector of the superlattice. The restriction on the 
interactions to the zigzag chains  and the use of simple-cubic supercells
modifies Eq. \ref{eq:periodic1} to the form,
\begin{equation}
  h_{i,j}=\sum_{\phi}\sum_{n=0}^{\infty}\frac{k}{\left(r_{i,j_{\phi}}+\sqrt{2}nL\right)^{\alpha}},
\label{eq:periodic2}
\end{equation}
where $\phi$ runs over all directions where the N atoms $i$ and $j$ are 
connected through a zigzag chain and $L$ is the side length of the cubic 
supercell. The diagonal 
terms become
\begin{equation}
  h_{i,i}=E_{s^*}+12\sum_{n=1}^{\infty}\frac{k}{\left(\sqrt{2}nL\right)^{\alpha}}.
\label{eq:diagonal}
\end{equation}
The inner sum in Eq.
\ref{eq:periodic2} is the Hurwitz zeta function and it can be evaluated
efficiently using the Euler-Maclaurin summation formula\cite{press}. 

To determine
the free parameters $k$ and $\alpha$, we created large ordered structures
corresponding to single N atom in 64-, 216-, 512-atoms supercells and two N 
atoms at the neighboring anion sites in the 216-atom supercell and fitted the 
parameters $k$ and $\alpha$ so that the TB eigenvalue distribution reproduces 
the characteristic features in our first-principles LDOS results shown in Figs.
\ref{fig:ldos}a-\ref{fig:ldos}b and \ref{fig:ldos_N2_near}. 
In the case of a single N atom the
fitted feature is the tail length and in the case of two N atoms the fitted
features are both the distance between the peaks corresponding to the bonding
and antibonding wave functions and the distance between their center of mass
and the conduction band minimum.
The optimal $k$ and $\alpha$ are found by searching for each value of 
$\alpha$ the optimal $k$ value  in the least squares sense.
The obtained parameters are $k=-0.67$ eV\r{A}$^\alpha$, $\alpha=1.28$ 
(GaAs$_{1-x}$N$_x$) and $k=-0.59$ eV\r{A}$^\alpha$, $\alpha=1.43$ 
(GaP$_{1-x}$N$_x$).

In the case of GaAs we checked the possibility that the N-N interactions are
not restricted to the zigzag chains and modelled them using the short-range
exponential decay. However, this model does not reproduce
the DFT results in Figs. \ref{fig:ldos}a-\ref{fig:ldos}b and 
\ref{fig:ldos_N2_near} and the error in the fit becomes nearly four times 
larger than in the case where the interaction are restricted on the zig-zag 
chains.

To simulate real structures, we distributed randomly from 384 up to 13,824 N 
atoms into a supercell of 442,368 anion sites. The resulting Hamiltonian matrix 
is diagonalized and a broadening eigenvalue distribution around the original 
energy level of nitrogen-induced states is obtained. 
Fig. \ref{fig:random_dist} shows the obtained 
distribution in the case of 13,824 N atoms. The side peaks around the 
nitrogen-induced peak results from the N pairs located at the 
neighboring anion sites.
%begin{figure*}[!t]
%  \centering
%    \subfloat[]{\epsfig{file=ldos_plots_GaAs.eps,width=0.4\linewidth,clip=}}
%    \hspace{1cm}
%    \subfloat[]{\epsfig{file=ldos_plots_GaP.eps,width=0.4\linewidth,clip=}}\\
%    \subfloat[]{\epsfig{file=ldos_plots_N2_near.eps,width=0.4\linewidth,clip=}}
%    \hspace{1cm}
%    \subfloat[]{\epsfig{file=opt_curves.eps,width=0.4\linewidth,clip=}}\\
%    \caption{(Color online) First-principles LDOS's at a N atom in 
%GaAs$_{1-x}$N$_x$ and GaP$_{1-x}$N$_x$ supercells (a)-(c) and least square error 
%in the fit as a function of the $\alpha$ parameter (d). In the case of a single 
%N atom in the supercell the solid 
%segment indicates the distance between the localized N 2$s$ state and the 
%conduction band minimum. In the case of a N pair the solid segments indicate
%the distance between the two N 2$s$ related peaks and the distance between the
%center of mass of the two peaks and the minimum eigenvalue. The energy zero
%coincides with the top of the valence band.}
%  \label{fig:ldos_supp}
%\end{figure*}
%\begin{figure*}[h]
%  \centering
%    \subfloat[]{\epsfig{file=dist_N2_near.eps,width=0.4\linewidth,clip=}}
%    \hspace{1cm}
%    \subfloat[]{\epsfig{file=random_dist.eps,width=0.4\linewidth,clip=}}\\
%    \caption{(Color online) TB eigenvalue distribution 
%(smoothed using a moving average) in the case of two N atoms in the 
%neighboring anion sites in the periodic GaAs$_{1-x}$N$_x$ structure (a) and TB
%eigenvalue distribution corresponding to a random sample of 13,824 N
%atoms in the GaAs$_{1-x}$N$_x$ alloy.}
%  \label{fig:simulations}
%\end{figure*}

\bibliography{references}

\end{document} 







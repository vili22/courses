\documentclass[aps,prb,10pt,twocolumn,groupedaddress]{revtex4-1}
%\setlength\topmargin{4.6mm}
%\documentclass{[prl,twocolumn]{revtex4-1}
%\usepackage[ansinew]{inputenc}
%\usepackage[latin1]{inputenc}% PROPER ENCODINGS
%\usepackage[T1]{fontenc}%      FOR FINNISH TEXT
%\usepackage[finnish]{babel}% 
%\usepackage[official]{eurosym}

%\usepackage{subfig}
\usepackage{caption}
\usepackage{subcaption}
\usepackage{graphicx}
\usepackage{epsfig}
\usepackage{epstopdf}
\usepackage{amsmath}
\usepackage{blkarray}
\usepackage{multirow}
\usepackage{mathtools}
\usepackage[font=small,labelfont=bf]{caption}
\input{mlbp17_macros}

%\usepackage{subfig}
%\usepackage[footnotesize]{caption}
%\pagestyle{empty}
%\setlength{\textwidth}{140mm}
%\setlength{\textheight}{240mm}
%\setlength{\parindent}{0mm}
%\setlength{\parskip}{3mm plus0.5mm minus0.5mm}
\bibliographystyle{apsrev4-1}
%%%%%%%%%%%%%%%%%%%%%%%%%%%%%%%%%%
\begin{document}

\title{Logistic regression and Bayes-classifier study of classification of
  songs to genres based on timbre, pitch and rhythm of the music signal }
\date{\today}
\author{John Doe}
\affiliation{Aalto University, P.O. Box 11100,
FI-00076 Aalto, Finland}

\begin{abstract}
  A genre of a song can be estimated based on its music signal's
  characteristics . In this work we use two classifiers, Bayes Classifier
  and Logistic classifier, to classify songs into one of ten possible genres. The
  two classifiers are trained against the training data and their performance is
  compared against each other. In this work we show that both classifiers
  perform much better compared to random guess. However their capability to
  classify all songs is clearly limited the accuracy for both classifiers being
  around 60\%.
\end{abstract}

\maketitle

\section{Introduction}
An automatic music transcription, \textit{i.e.}, notating a piece of music to
a speficic genre, \textit{e.g.}, Blues, dates back to 1970s when first attempts
towards automatic music transcription were made\cite{musictranscription}. Since
then interest in
automatic transcription of music has grown rapidly and various approaches,
statistical methods, modelling human auditory system, have been applied to
music transcription problem. However even today an expert human musician often
beats a state-of-the-art automatic transcription system in accuracy.

Characteristics of music signal that are useful in classification of a song
are \textit{timbre}, \textit{rhythm}, \textit{pitch}, \textit{loudness} and
\textit{duration}\cite{musictranscription} from which the three first one,
described below are used in this work.
\begin{itemize}  
\item The timbre of the music can be most easily described as the factor
  which separates two sources of music from each other. For example if the same
  song is played by violin or a guitar the timbre is called the character which
  separates the violin from the guitar.
\item The pitch is related to frequency scale of a song a can be defined as the
  frequency of the sine-wave fitted to target sound by human listener.
\item The rhythm of the music can be described as arrangement of sounds as time
  flows.
\end{itemize}

In classification problem the object is classified into a certain class based
on it's characteristics called features. A linear classifier does the
classification by making a linear combination of the features and converting
the resulting value into a class or a probability that the object belongs to
given class. In logistic regression the feature vector of the object is
transformed into a probality by taking a linear combination of features and
mapping the result into interval $\lbrack 0, 1\rbrack$ using a sigmoid function.
The Bayes-classifier in contrast assumes that the feature vector is drawn from a
multidimensional-Gaussian distribution. The posterior probability of the object
belonging to a certain class is then obtained as a product of the prior of the
class and the probability to sample the given feature vector from the
multidimensional Gaussian distribution.

The paper is organized as follows. The used data-set and the computational
methods are described in detail in Sec. \ref{sec:methods}. In Sec.
\ref{sec:results} the results for the both logistic regression- and
Bayes-classifier are given. Sec. \ref{sec:conclusions} is a
summary of the results and the differences between the two classifiers are
discussed.

\section{Used data-set and computational methods}
\label{sec:methods}
\subsection{Used data-set}
\label{sec:used_data_set}
The data-set consisted of 4363 songs and was divided into training and test data
sets including every third song to test set and rest of the songs to training
set. Each song contained 264 features and the songs were labeled to 10
different categories. The gatecories were: 1 Pop Rock, 2 Electronic, 3 Rap,
4 jazz, 5 Latin, 6 RnB, 7 International, 8 Country, 9 Reggae and 10 Blues.
The musical characteristics of the songs were packed to a feature
vector of length 256. The first 48 elements in the feature vector can be
associated to timbre, the next 48 elements to pitch and the final 168 features
to rhythm. The distribution of the features resembled in most cases a Gaussian
distribution or a skew symmetric distribution. This is illustrated figures
\ref{fig:feature_distribution}a and \ref{fig:feature_distribution}b.
\begin{figure}[!t]
  \centering
  \begin{subfigure}[]{0.38\textwidth}
    \centering
    \includegraphics[width=\textwidth]{feature_gaussian.eps}
    \caption{}
  \end{subfigure}\\
  %\vspace{0.1cm}
  \centering
  \begin{subfigure}[]{0.38\textwidth}
    \centering
    \includegraphics[width=\textwidth]{feature_skew_symmetric.eps}
    \caption{}
  \end{subfigure}\\
  \caption{Visualization of typical distributions of features, Gaussian distribution (a) and skew-symmetric distribution (b).}
  \label{fig:feature_distribution}
\end{figure}

\subsection{Computational methods}
\label{sec:computational_methods}
In this work two different methods were used to classify the songs to different
genres. First method is logistic-regression method in which the logistic-loss
is minimized iteratively using the gradient descent method.
The other method used is the Bayes-classifier which classifies the song to
certain category that gives the maximum posterior probability with respect to
label $i$. Both methods are described below in detail. In addition we studied
the effect of feature extraction and for that purpose we used principal
component analysis method to exclude features with little impact. 
\subsubsection{Logistic-regression}
\label{sec:gradient_descent}
In logistic regression for binary classifier problem the starting point is the
minimization of the loss function
\vspace*{-2mm}
 $$\emperror((\vx,y);\vw) =\min\limits_{\vw \in \mathbf{R}^{2}} (1/\samplesize)\sum_{\sampleidx=1}^{\samplesize} L((\vx^{(\sampleidx)},y^{(\sampleidx)}); \vw),$$
where the logistic loss $L((\vx,y); \vw)$ is defined as   $L((\vx,y); \vw) = \ln\big(1 + \exp\big(- y (\vw^{T} \vx)\big))\big)$ and
$\vx$ is the feature vector of a music signal, $\vw$ are the coefficients of the linear expansion and $y$ is the label 1 or -1 whether the
song belongs to certain category or not. The minimization problem can be further converted to
\begin{eqnarray}
  \emperror((\vx,y);\vw)&=&\frac{1}{N}\max\limits_{\vw \in \mathbf{R}^{2}}\sum_{i=1}^{N}\mathrm{ln}(\frac{1}{1+\mathrm{exp}(-y^i(\vw^T\vx))})\nonumber\\
  &=&\frac{1}{N}\max\limits_{\vw \in \mathbf{R}^{2}}(\sum_{y^i=1}\mathrm{ln}(\frac{1}{1+\mathrm{exp}(-\vw^T\vx^{(y^i)})})\nonumber\\
  & & + \sum_{y^i=-1}\mathrm{ln}(1-\frac{1}{1 + \mathrm{exp}(-\vw^T\vx^{(y^i)})}))\\
  &=&\max\limits_{\vw \in \mathbf{R}^{2}}(\sum_{y^i=1}\mathrm{ln}(p_{y^i=1})\nonumber\\
  & & + \sum_{y^i=-1}\mathrm{ln}(1-p_{y^i=1})),
  \label{eq:logreg1}
\end{eqnarray}
where $p_{y^i=1}$ is the probability that the song $i$ is labeled belonging to certain category.
There is no closed form solution for equation (\ref{eq:logreg1}) and for that reason some numerical iterative solver must
be used to find the optimal $\vw$. One of the most popular methods to find the optimal solution is gradient descent (GD) method.
In GD the weights $\vw$ are updated at each iteration $\mathrm{k+1}$ according to equation
\begin{equation}
  \vw^{(k+1)} = \vw^{(k)} - \alpha \nabla \emperror(\mathbf{w}^{(k)}).
  \label{eq:gd}
\end{equation}

To be able to use the GD method we need know the gradients  $\nabla \emperror(\mathbf{w}^{(k)})$.
By marking $t^i=\mathrm{max}(0,y^i)$ we can write the equation (\ref{eq:logreg1}) as
\begin{eqnarray}
  \emperror((\vx,y);\vw)&=&\min\limits_{\vw \in \mathbf{R}^{2}}-\sum_i^N(t^i\mathrm{ln}(p_{y^i})\nonumber\\
  & &+(1-t^i)\mathrm{ln}(1-p_{y^i})).
  \label{eq:logreg2}
\end{eqnarray}
The derivative of $p_{y^i}$ with respect to $w_j$ can be written as $\frac{\partial p_{y^i}}{\partial w_j}=\frac{\partial p_{y^i}}{\partial (\vw\vx)}\frac{\partial (\vw\vx)}{\partial w_j} = p_{y^i}(1-p_{y^i})x_j^i$. Thus taking derivative of equation (\ref{eq:logreg2}) with
respect to $w_j$ we get for $\nabla \emperror(w_j^k)$.
\begin{eqnarray}
  \nabla \emperror(w_j^k) &=&-\sum_i(t^i\frac{1}{p_{y^i}}(1-p_{y^i})p_{y^i}x_j^i\nonumber\\
  & &+(1-t^i)\frac{1}{1-p_{y^i}}(-1)p_{y^i}(1-p_{y^i})x_j^i)\nonumber\\
  &=&-\sum_i\left(t^i -t^ip_{y^i}-p_{y^i}+t^ip_{y^i}\right)x_j^i\nonumber\\
  &=&-\sum_i(t^i-p_{y^i})x_j^i.
\end{eqnarray}
Now the $\nabla \emperror(w_j^k)$'s can be written in vector form as
\begin{equation}
  \nabla \emperror(\mathbf{w}^k) = -\left(\hat{\vy}-\frac{1}{1+\mathrm{exp}(-(\vw^{k})^T X^T)}\right)^TX,
  \label{eq:gd_gradients}
\end{equation}
where $\hat{\vy} = \mathrm{max}(\mathbf{0},\vy)$ is a vector of length $N$ for which the
$i$-th element is zero if $y_i=-1$ and one if $y_i$ = 1. The matrix $X$ is a
matrix which rows are the feature vectors $\vx^i$. Thus we can solve the
minimization problem (\ref{eq:logreg2}) iteratively using the GD
method (\ref{eq:gd}) and the gradients (\ref{eq:gd_gradients}).
The parameters $\vw$ are solved for each class $i$ using the training data. The
category $i$ of a song in test-set is then the one with the largest probability
$p(y_i)$.
\subsubsection{Bayes classifier}
\label{sec:bayes_classifier}
For a Bayes classifier we assume that the distribution of the feature vector of
a music signal with respect to label $y_i$ is a Gaussian distribution
\begin{equation}
  p(\vx|y_i;\mathbf{m}_{i},\mathbf{C}_{i}) = \frac{1}{\sqrt{\det\{ 2 \pi \mathbf{C}_{i} \}}}e^{-(1/2) (\vx\!-\!\vm_{i})^{T} \mathbf{C}_{i}^{-1} (\vx\!-\!\vm_{i})}.
  \label{eq:bayes_gaussian_distribution}
\end{equation}
Using the Baye's theorem the posterior probability
$p(y_i|\vx;\mathbf{m}_{i},\mathbf{C}_{i})$ can be written as
\begin{equation}
  p(y_i|\vx;\mathbf{m}_{i},\mathbf{C}_{i}) = \frac{p(y_i)p(\vx|y_i;\mathbf{m}_{i},\mathbf{C}_{i})}{p(\vx)},
  \label{eq:bayes_posterior}
\end{equation}
where the $p(\vx)$ is a normalization constant and can be omitted. To be able to
use the equation (\ref{eq:bayes_posterior}) we need to find optimal values for
parameters $p(y_i)$, $\vm_i$ and $\mathbf{C}_i$. The prior $p(y_i)$ can be
simply estimated as the fraction of labels $y_i$ among all labels.
Because the samples ${\vx_i^{(t)}, y_i^{(t)}}$ are independent the parameters
$\vm_i$ and $\mathbf{C}_i$ and can be obtained by
maximizing the respective log-likelihood function with respect to the
parameters. The log-likelihood
can be written as
\begin{eqnarray}
  \mathcal{L}_{y=i}&=&\sum_{y^t=i}\mathrm{log}(P(\vx^t|y^t=i;\vm_i,\mathbf{C}_i))\nonumber\\
  &=&\sum_{y^t=i}(-\frac{1}{2}\mathrm{log}(2\pi^n)-\frac{1}{2}\mathrm{log}(\mathrm{det}(\mathbf{C}_i))\nonumber\\
  & &-\frac{1}{2}(\vx^t-\vm_i)^T\mathbf{C}_i^{-1}(\vx^t-\vm_i)).
\end{eqnarray}
Now the optimal $\vm_i$ is obtained by setting the derivative of
$\mathcal{L}_{y=i}$ with respect to $\vm_i$ to zero, \textit{i.e.},
\begin{eqnarray}
  \frac{\partial{\mathcal{L}_{y=i}}}{\partial\vm_i}&=&\mathbf{C}_i^{-1}\sum_{y^t=i}(\vx^t-\vm_i).
  \label{eq:bayes_ms}
\end{eqnarray}
Setting eq. (\ref{eq:bayes_ms}) to zero and solving for $\vm_i$ gives
$\vm_i=\frac{\sum_{y^t=i}\vx^t}{N}$. Similarly the
$\mathbf{C}_i$ is obtained  by setting the derivative of $\mathcal{L}_{y=i}$ with respect to $\mathbf{C}_i$ to zero giving
\begin{eqnarray}
  \frac{\partial{\mathcal{L}_{y=i}}}{\partial\mathbf{C}_i}&=&\sum_{y^t=1}(-\frac{1}{2}\mathbf{C}_i^{(-1)}\nonumber\\
  & &+\frac{1}{2}\mathbf{C}_i^{(-1)}(\vx^t-\vm_i)(\vx^t-\vm_i)^T\mathbf{C}_i^{-1}).
  \label{eq:bayes_cs}
\end{eqnarray}
Setting eq. (\ref{eq:bayes_cs}) to zero and solving for $\mathbf{C}_i$ gives
$\mathbf{C}_i=\frac{1}{N}\sum_{y^t=i}(\vx^t-\vm_i)(\vx^t-\vm_i)^T$. In equations
(\ref{eq:bayes_ms}) and (\ref{eq:bayes_cs}) the identities
$\frac{\partial(\vx-\vs)^T\mathbf{W}(\vx-\vs)}{\partial\vs} = 2\mathbf{W}(\vx-\vs)$, $\frac{\partial\mathrm{ln}|\mathrm{det}(\mathbf{X})|}{\partial\mathbf{X}}=(\mathbf{X}^T)^{-1}$ and  $\frac{\partial\mathbf{a}^T\mathbf{X}^{-1}\vb}{\partial\mathbf{X}}=-\mathbf{X}^{-T}\mathbf{a}\vb^T\mathbf{X}^{-T}$ from matrix cook-book
\cite{matrixcookbook}
were used. In classification the parameters $p(y_i)$, $\vm_i$ and $\mathbf{C}_i$
are optimized for all ten classes using the training data. The song is then
classified to certain category $i$ that maximizes the posterior probability
(\ref{eq:bayes_posterior}), \textit{i.e.},
\begin{equation}
  i = \argmax_i p(y_i|\vx;\mathbf{m}_{i},\mathbf{C}_{i}).
\end{equation}
\subsubsection{Principal component analysis}
\label{sec:principal_component}
Let $\mathbf{X}$ be matrix which rows are the feature vectors, \textit{i.e.},
the dimension of the matrix is $n\times p$ where $n$ is the number of samples
and p is the length of the feature vector.
The sample covariance matrix $\mathbf{C}$ is then obtained as
$\mathbf{C}=\mathbf{X}^T\mathbf{X}/(n-1)$ which can be diagonalized as
\begin{equation}
  \mathbf{C}=\mathbf{V}\mathbf{L}\mathbf{V}^T,
  \label{eq:c_diagonalization}
\end{equation}
where $\mathbf{V}$ are the eigenvectors of $\mathbf{C}$ and are called
the principal axes.
The matrix $\mathbf{X}$ can be decomposed using singular-value decomposition as
$\mathbf{X}=\mathbf{U}\mathbf{S}\mathbf{V}^T$, where $\mathbf{S}$ is a diagonal
matrix containing the singular values of $\mathbf{X}$. Now the matrix
$\mathbf{C}$ can be written as
\begin{equation}
  \mathbf{C} = \mathbf{V}\frac{\mathbf{S}^2}{n-1}\mathbf{V}^T.
  \label{eq:c_svd}
\end{equation}
Comparing equation (\ref{eq:c_svd}) to (\ref{eq:c_diagonalization}) shows that
the principal axes are the same as the right singular vectors of $\mathbf{X}$.
Now the principal components of $\mathbf{X}$ are obtained as
$\mathbf{X}\mathbf{V}=\mathbf{U}\mathbf{S}$. Now we can include only feature
vectors that have impact by excluding components that corresponds to singular
values below some threshold. In this work we included singular values of which
sum contained 90\% of the total sum of the singular values and excluded rest.
\section{Results}
\label{sec:results}
Both the logistic-regression classifier and the Bayes classifier where trained
using the training data set. After that the trained classifiers were applied to
test data set and the accuracy and the logarithmic-loss of the classifiers were
evaluated. In addition to test data set an external feature set with unknown
labels were classified and the accuracy and the logarithmic-loss of classifiers
were evaluated on an external server and results were compared to those obtained
from the used test data set. The accuracy of the classifier was simply evaluated
as the fraction of correct labels with respect to the total number of labels
$accuracy = \frac{|y_{true}=y_{predicted}|}{N}$. The logarithmic loss was
evaluated as $log-loss = 1/N\sum_{i=1}^N\sum_{j=1}^My_{ij}log(p_{ij})$ where
$y_{ij}$ is and indicator function evaluated to 1 if sample $i$ was labeled to
class $j$ and zero otherwise and the $p_{ij}$ is the corresponding class
probability. The use logarithm  is the base 10 logarithm.
\subsection{Logistic-regression}
\label{sec:result_logistic_regression}
The accuracy obtained using the logistic-regression and all elements of the
feature vectors was 0.66 and the corresponding logarithmic-loss was 0.27. The
accuracy and the logarithmic loss evaluated on the external server were 0.65 and
0.178 respectively. We also tested to use principal-component analysis to reduce
the number of features. The corresponding accuracy and logarithmic-loss were
0.64 and 0.30 respectively being somewhat worse than without feature
extraction. However it is good to notice that with feature extraction the
computation was significantly faster. In table \ref{tab:confusion_log_reg} the
confusion matrix corresponding to full feature vector classification is shown.
\begin{center}
  \begin{table}
    \caption{Confusion matrix corresponding to classification obtained using
      logistic regression. The column direction indicates the true value and
      the row direction is the predicted value.  The labels $1\ldots10$ are
      the ten music genres specified in section \ref{sec:used_data_set}.}
    \begin{tabular*}{0.45\textwidth}{@{\extracolsep{\fill}}c|cccccccccc}
        & 1 & 2 & 3 & 4 & 5 & 6 & 7 & 8 & 9 & 10\\
      \hline
      1 & 652 & 35 & 7 & 9 & 4 & 10 & 1 & 3 & 2 & 3\\
      2 & 57 & 130 & 9 & 4 & 2 & 1 & 0 & 2 & 2 & 0\\
      3 & 14 & 9 & 82 & 4 & 2 & 1 & 0 & 0 & 1 & 0 \\
      4 & 25 & 3 & 0 & 43 & 0 & 2 & 0 & 1 & 0 & 2 \\
      5 & 38 & 4 & 1 & 5 & 11 & 1 & 1 & 0 & 3 & 0 \\
      6 & 37 & 6 & 12 & 8 & 4 & 25 & 0 & 0 & 0 & 0 \\
      7 & 34 & 6 & 1 & 2 & 2 & 4 & 3 & 1 & 0 & 0 \\
      8 & 50 & 0 & 0 & 1 & 2 & 1 & 0 & 9 & 0 & 0 \\
      9 & 2 & 5 & 0 & 2 & 1 & 0 & 0 & 8 & 0 & 27 \\
      10 & 21 & 0 & 0 & 6 & 1 & 2 & 0 & 1 & 0 & 3\\
      \end{tabular*}
    \label{tab:confusion_log_reg}
  \end{table}
\end{center}

\subsection{Bayes classifier}
\label{sec:result_bayes_classifier}
For Bayes-classifier the sample covariance matrix $C_i$ defined in section
\ref{sec:bayes_classifier} becomes singular if all elements in the feature
vector are used and for that reason classification was only performed alongside
with the feature extraction. In feature extraction we selected features
corresponding to singular values that counted 80\% of the total sum of
singular values and excluded rest. The obtained accuracy was 0.53 and the
corresponding logarithmic-loss was 0.33. The accuracy and the logarithmic loss
evaluated on the external server were 0.32 and 1.17 respectively significantly
lower performance than with the test data. In table \ref{tab:confusion_bayes}
the confusion matrix corresponding to Bayes-classifier classification is shown.
\begin{center}
  \begin{table}[b]
    \caption{Confusion matrix corresponding to classification obtained using
      Bayes-classifier. The column direction indicates the true value and the
      row direction is the predicted value. The labels $1\ldots10$ are the ten
    music genres specified in section \ref{sec:used_data_set}.}
    \begin{tabular*}{0.45\textwidth}{@{\extracolsep{\fill}}c|cccccccccc}
        & 1 & 2 & 3 & 4 & 5 & 6 & 7 & 8 & 9 & 10\\
      \hline
      1 & 544 & 172 & 0 & 7 & 0 & 2 & 0 & 1 & 0 & 0\\
      2 & 27 & 174 & 3 & 2 & 0 & 0 & 0 & 1 & 0 & 0\\
      3 & 9 & 53 & 51 & 0 & 0 & 0 & 0 & 0 & 0 & 0 \\
      4 & 20 & 48 & 0 & 6 & 0 & 1 & 0 & 1 & 0 & 0 \\
      5 & 31 & 32 & 0 & 0 & 0 & 0 & 0 & 1 & 0 & 0 \\
      6 & 28 & 57 & 5 & 0 & 0 & 2 & 0 & 0 & 0 & 0 \\
      7 & 21 & 31 & 1 & 0 & 0 & 0 & 0 & 0 & 0 & 0 \\
      8 & 53 & 8 & 0 & 1 & 0 & 0 & 0 & 1 & 0 & 0 \\
      9 & 6 & 18 & 3 & 0 & 0 & 0 & 0 & 0 & 0 & 27 \\
      10 & 15 & 18 & 0 & 0 & 0 & 0 & 0 & 1 & 0 & 0\\
      \end{tabular*}
    \label{tab:confusion_bayes}
  \end{table}
\end{center}
\section{Conclusions}
\label{sec:conclusions}
In this work we used logistic-regression and Bayes-classifier to classify songs
to different genres based on the music signal's characteristics. For logistic
regression the obtained accuracy for test set was 0.67 and logistic-loss 0.27.
For the external data set used the obtained accuracy and logistic-loss were 0.65
and 0.178 respectively in the case of logistic-regression. For the
Bayes-classifier the obtained accuracy and logistic-loss were 0.53 and 0.33 for
the test-data and for external data set 0.32 and 1.17 respectively. According to
obtained results both classifiers performed clearly better than random guess,
but remained far from perfect classification. From the two classifiers used
the logistic-regression classifier performed clearly better. The logistic
classifier also generalized much better to completely new data giving nearly
equal performance for test data set and external data set.

\bibliography{references}

\end{document} 






